% Generated by GrindEQ Word-to-LaTeX 
\documentclass{article} % use \documentstyle for old LaTeX compilers

\usepackage[english]{babel} % 'french', 'german', 'spanish', 'danish', etc.
\usepackage{amssymb}
\usepackage{amsmath}
\usepackage{txfonts}
\usepackage{mathdots}
\usepackage[classicReIm]{kpfonts}
\usepackage{graphicx}

% You can include more LaTeX packages here 


\begin{document}

%\selectlanguage{english} % remove comment delimiter ('%') and select language if required


\noindent Normal

\noindent 

\noindent 

\noindent 

\noindent jr istruzione che rid\`{a} controllo a quello che ha chiamato

\noindent \$ra = Return Address, \`{e} l'indirizzo di ritorno 

\noindent .data indica che tutti i fati sono storati in questa zona

\noindent .text indica la sezione in cui so  no presenti le istruzioni nella sezione di testo

\noindent .globl/.global \`{e} una direttiva che il simbolo \`{e} accessibile da fuori dal file (non \`{e} scritto dentro)

\noindent .word stora gli n valori in 32 bit in w1, w2, {\dots} , wn parole consecutive di memoria

\noindent .half come word, ma stora in 16 bit h1, {\dots} , hn parola `` ``

\noindent .space alloca uno spazio di n byte nel seg

 es: .globl main  definisce globalmente main

\noindent la vuol dire Load Address, ed \`{e} seguito da un registro e un valore 

 es: la \$t0, X carica l'indirizzo di X in \$t0

\noindent 

\noindent 

\noindent All'inizio di ogni programma, il simulatore aggiunge:

\noindent     lw    \$a0 0(\$sp)   \# argc 

\noindent     addiu \$a1 \$sp 4    \# argv 

\noindent     ...

\noindent     jal   main 

\noindent     nop 

\noindent     li    \$v0 10 

\noindent     syscall            \# syscall 10(exit) 

\noindent 

\noindent che sarebbe il codice di inizializzazione, tipo il main, che setta argc e argv

\noindent \# Name and general description of program 

\noindent \# ---------------------------------------- 

\noindent \# Data declarations go in this section.

 .data 

\noindent \# program specific data declarations 

\noindent \# ---------------------------------------- 

\noindent \# Program code goes in this section. 

\noindent .text 

  .globl  main 

  .ent  main 

\noindent main: 

\noindent \# ----- 

\noindent \# your program code goes here. 

\noindent \# ----- 

\noindent \# Done, terminate program. 

li \$v0, 10 syscall  \# all done!

 .end main 

\noindent 

\noindent 

\noindent The initial header (".text", ".globl main", ".ent main", and "main:") will be the same for all QtSpim programs. The final instructions ("li \$v0, 10" and "syscall") terminate the program.

\noindent 

\noindent 

\noindent nome reg num reg descrizione

\noindent \$zero   0   costante zero 

\noindent \$at   1   riservato per l'assemblatore 

\noindent \$a0-\$a3  4-7   argomenti di una procedura 

\noindent \$gp   28   global pointer alla global area (dati) 

\noindent \$ra   31   indirizzo di ritorno

\noindent \$s0-\$s7  16-23   registri salvati 

\noindent \$s8   30   registro salvato (fp) 

\noindent \$t0-\$t7   8-15   registri temporanei (non salvati) 

\noindent \$t8-\$t9   24-25   registri temporanei (non salvati)

\noindent \$v0-\$v1  2-3   valori di ritorno di una procedura 

\noindent \$sp   29   stack pointer 

\noindent \$k0-\$k1   26-27   gestione delle eccezioni 

\noindent Aritmetica mips

\noindent 

\noindent Operatori add e sub  (addizione e sottrazione)

 Sintassi: add \$1 ,\$2, \$3      assegna come valore di \$1 la somma di \$2 + \$3

   sub \$1 ,\$2, \$3      assegna come valore di \$1 la differenza di \$2 - \$3

\noindent es.

\noindent main:

 li \$t0, 25  \#assegno valore 25 a t0

 li \$t1, 10  \#assegno valore 10 a t1

 add \$t2, \$t0, \$t1 \#storo la somma di t0 e t1 in t2

 li \$v0, 10  \#fine programma, 10 \`{e} il codice per l'uscita{\dots}

 syscall   \# {\dots} dal programma viene messo in v0

\noindent 

\noindent System calls (chiamate a sistema)

\noindent \includegraphics*[width=4.98in, height=5.58in, keepaspectratio=false]{image1}


\end{document}

